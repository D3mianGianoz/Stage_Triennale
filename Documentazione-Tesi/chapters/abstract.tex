

\thispagestyle{plain}
\begin{center}
	\Large
	\textbf{Diagnosis by Numbers}
	
	\vspace{0.4cm}
	\large
	Uno strumento basato sulle logiche descrittive, "tipicalità" e probabilità
	
	\vspace{0.4cm}
	\textbf{Damiano Gianotti}
	
	\vspace{0.9cm}
	\textbf{Abstract}
\end{center}

Lo stage si pone l’obiettivo di realizzare \textbf{DbN}, un tool di supporto per la \textit{diagnosi differenziale}, 
con applicazione ad un caso di studio in ambito medico, basato su una logica descrittiva con tipicalità e probabilità di avere eccezioni.
La logica in questione consente di generare scenari plausibili ma “sorprendenti”, che consentiranno di formulare
diagnosi non ovvie  (iter alternativi potenzialmente rilevanti) e di stimarne la probabilità. 
Lo strumento potrebbe portare ad ulteriori ricerche, nel caso in cui le spiegazioni
più plausibili non siano quelle corrette e mostrare possibili scenari non banali.
 
Il lavoro iniziale è stato lo studio delle Logiche Descrittive, una famiglia 
di linguaggi formali utilizzati per esprimere (rappresentare) la conoscenza 
in un dominio specifico (detto mondo). Sono quindi alla base dei linguaggi impiegati per lo sviluppo
di ontologie nel Web Semantico, come il Web Ontology Language (OWL).

In seguito, il progetto è partito come estensione della tesi \citetitle{PEAR} \cite{PEAR}. Infatti, dopo averne studiato le caratteristiche importanti, si è cercato di costruire sopra
un diverso sistema di generazione degli scenari e di rafforzarne le componenti di ragionamento, fortemente orientati all'implementazione dell'esempio di \citetitle{ProbOfEx} \cite{ProbOfEx}.
Raggiunto questo obiettivo, non banale, si è ottimizzato e pulito il codice
e sono state aggiunte ulteriori importanti funzionalità, come la creazione di grafici interattivi e costi
diagnostici.

\paragraph{Struttura della tesi} \hfill

Di seguito, il piano di quest'opera.
\begin{itemize}
	\item Il primo capitolo dà una breve infarinatura sulle fondamenta del progetto e descrive schematicamente il lavoro che è stato svolto.
	\item Il secondo capitolo invece fornisce alcune tra le nozioni teoriche più importanti che
	servono per comprendere i meccanismi su cui si basa DbN
	\item Il terzo tratta di quali librerie/linguaggi son state/i utilizzate/i e il perché
	\item Il quarto descrive, in dettaglio, le singole componenti del \textit{software}
	\item Il quinto racconta dei principali problemi presenti e delle possibili idee risolutive
\end{itemize}