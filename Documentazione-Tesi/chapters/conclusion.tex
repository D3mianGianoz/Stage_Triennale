Questo breve capitolo finale serve come resoconto di tutto quello
presentato fin’ora e cerca di dare un’idea su come questo tool 
potrebbe essere impiegato, e sulle possibili future evoluzioni che potrà assumere.

Il focus del progetto è il supporto alle decisioni del medico,  l'affiancamento, non la sostituzione di tale figura professionale,

Il considerare i sintomi, normalmente catalogati come atipici, per una certa malattia, 
sotto una diversa luce potrebbe permettere di scoprire che, in realtà, sono indirettamente collegati ad altre patologie.
Per rendere l’idea si pensi ad una persona che soffre di obesità; tendenzialmente 
un obeso ha problemi di metabolismo, scarsa autostima e alti livelli di colesterolo. 
L’utilizzo di smartphone durante i pasti è stato riscontrato essere un fattore incidente sull’aumento di peso, poiché i pazienti testati, tendono a prestare meno attenzione a quello che mangiano e, come conseguenza, ingeriscono maggiori quantità di cibo. 
Questo fatto, indirettamente collegato con l’obesità, potrebbe essere, in alcuni soggetti,
un fattore decisivo. Come conseguenza, questa situazione, se ben rappresentata nella KB, potrebbe esser modellata nello strumento come uno scenario poco probabile, ma sempre possibile; la chiave è avere sottomano lo spettro completo delle alternative.

Per quanto riguarda possibili migliorie, le strade percorribili sono numerose; a partire dall'interfaccia utente: la creazione di una GUI e la semplificazione del processo 
di immissione dati rappresenterebbe un notevole passo avanti per quanto riguarda l'usabilità
e abbasserebbe la curva di apprendimento del software.
Pensando, invece, all'effettivo test ed utilizzo sul campo, è necessario un lavoro di studio,
modifica e creazione di ontologie realistiche o semi-realistiche per dare una parvenza di 
veridicità alle diagnosi prodotte, soprattutto per quanto riguarda le probabilità delle
relazioni e i costi delle diagnosi.
Un altro  possibile percorso potrebbe essere quello di computare e aggiungere ulteriori metadati ai singoli scenari, come, ad esempio, tempistiche indicative e/o un elenco di esami/visite necessarie alla verifica della "diagnosi".
Se dovessimo pensare a delle ottimizzazioni e perfezionamenti, una delle prime idee possibili è l'utilizzodi super-classi \textbf{OWL} più significative del generico \mintinline{XML}{Thing}; classi come
\mintinline{XML}{Patient}, \mintinline{XML}{MedicalIllness} o \mintinline{XML}{Symptom} scritte in un ontologia di supporto.
